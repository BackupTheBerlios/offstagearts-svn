\documentclass[11pt]{article}

\setlength{\topmargin}{-.5in}
\setlength{\textheight}{9in}    % 9 inch page
\setlength{\textwidth}{5.5in}     % 5 3/4 inch line 
\setlength{\oddsidemargin}{36pt}  % 1/4 inch (yields 1 1/4 inch left margin)
\setlength{\evensidemargin}{36pt} % 1/4 inch

\title{OffstageArts User Manual (draft)}

\author{Robert Fischer}
\usepackage{float,graphics,url,times,listings}

%\define{\emph{OffstageArts}}{\emph{OffstageArts}}

\begin{document}

\maketitle

\section{Creating a new Database}

Creating a new OA database is pretty easy.  Only the database needs to be created; OA will create the appropriate schema.  Make sure you are using PostgreSQL version 8 or higher.  You merely need to run the script below (or something equivalent to it).  Note that this script will create a PostgreSQL database and user of the same name.

\begin{verbatim}
#!/bin/sh
# Usage: newuser name password

psql template1 -U postgres <<EOF
CREATE ROLE $1 LOGIN PASSWORD '$2'
   VALID UNTIL 'infinity'
   CONNECTION LIMIT 50;

CREATE DATABASE $1
  WITH ENCODING='UTF8'
       OWNER=$1;
EOF

psql $1 -U postgres <<EOF
CREATE PROCEDURAL LANGUAGE plpgsql;
ALTER SCHEMA public OWNER TO $1;
EOF
\end{verbatim}

Once a new database is created, OA needs to be configured to connect to it.  Th e first time it connects, it will generate the appropriate schema.

\section{Connection Configuration}

In order to configure OA to connect to a specific database:

\begin{enumerate}

 \item Launch OA via the class \emph{offstagearts.gui.MainLauncher} (or just run it off the JNLP file from the website).

 \item In the configuration screen, click on ``Setup.''

 \item In the Setup scren, click ``New.''

 \item Enter any name you like under ``Name.'' Choose the name of a new directory under ``Directory.''  This will be your configuration directory.  When you're done choosing these items, click ``OK.''

 \item You now see a window called ``Edit Configuration.'' It allows you to edit the four propreties files that were created in your configuration directory.  The ``All'' tab is properties that are the same on all platforms.  You should alos edit anything under the ``Mac,'' ``Windows'' or ``Linux'' tab, depending on your platform.

 \item After the configuration has been edited, you should be able to launch that configuration from the main configuration screen.

\end{enumerate}

\end{document}
